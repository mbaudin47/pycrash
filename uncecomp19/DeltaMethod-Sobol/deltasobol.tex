% Michaël Baudin, 2017
\documentclass{article}

\usepackage[utf8]{inputenc}
\usepackage[T1]{fontenc}

\usepackage{lmodern} % d'autres polices sont possibles

\usepackage[a4paper]{geometry}

\usepackage{authblk}
\usepackage{url}

\usepackage{graphicx}
\DeclareGraphicsExtensions{.jpg,.pdf,.png}

\newcommand{\ot}{Open TURNS}

\newcommand{\pyobj}[1]{\texttt{#1}}

\usepackage{listings}

\usepackage{amsmath}
\usepackage{amsthm}

\usepackage[numbers]{natbib}

% Double barred letters
\usepackage{amssymb}
\newcommand{\RR}{\mathbb{R}}
\newcommand{\bx}{{\bf x}}
\newcommand{\by}{{\bf y}}
\newcommand{\bg}{{\bf g}}
\newcommand{\bX}{{\bf X}}
\newcommand{\bzero}{{\bf 0}}
\newcommand{\btheta}{\boldsymbol{\theta}}

\newtheorem{definition}{Definition}
\newtheorem{example}{Example}
\newtheorem{theorem}{Theorem}
\newtheorem{remark}{Remark}


%%%%%%%%%%%%%%%%%%%%%%%%%%%%%%%%%%%%%%%%%%%%%%%%%%%%%%%%%%%%%%%%%%%

\begin{document}


\title{The Delta-Method applied to Sobol' indices}

\author{Michaël Baudin}

\maketitle


\abstract{
We explore the use of the Delta-method in order to estimate 
the Sobol' indices.
}

\tableofcontents

%%%%%%%%%%%%%%%%%%%%%%%%%%%%%%%%%%%%%%%%%%%%%%%%%

\section{Convergence in distribution}

\begin{definition}
(\emph{Convergence in distribution})
Assume that $X_1,X_2,...$ is a sequence of real-valued random variables 
with cumulative distribution functions $\{F_n\}_{n\geq 0}$. 
Assume that $X$ is a real-valued random variable 
with cumulative distribution function $\{F\}_{n\geq 0}$. 
The sequence $X_n$ converges in distribution to $X$ if:
$$
\lim_{n\rightarrow \infty} F_n(X_n)=F(x).
$$
for any $x\in \RR$ at which $F$ is continuous.
In this case, we write:
$$
X_n \xrightarrow{D} X.
$$
\end{definition}

The following theorem gives an example of such convergence. 

\begin{theorem}
(\emph{Maximum of uniform random numbers})
Assume that $X_1,X_2,...$ are independent random numbers such that $X_n\sim U(0,1)$. 
Let $Y_n$ be the maximum:
$$
Y_n = \max_{1\leq i\leq n}  X_i.
$$
Therefore the sequence $Y_n$ converges in distribution to an exponential random variable, i.e.:
$$
n(1-Y_n) \xrightarrow{D} \mathcal{E}(1).
$$
\end{theorem}

\begin{proof}
\end{proof}


%%%%%%%%%%%%%%%%%%%%%%%%%%%%%%%%%%%%%%%%%%%%%%%%%

\section{Delta method}

\begin{theorem}
(\emph{Delta-method})
Assume that $X_1,X_2,...$ is a sequence of real-valued random variables 
so that 
$$
\sqrt{n} (X_n - \theta) \xrightarrow{D} \mathcal{N}(0,\sigma^2).
$$
Assume that $g$ is a real function. 
Let $\theta\in\RR$ and suppose that $g'(\theta)$ exists and that $g'(\theta)\neq 0$. 
Therefore, 
$$
\sqrt{n} (g(X_n) - g(\theta)) \xrightarrow{D} \mathcal{N}(0,\sigma^2 g'(\theta)^2).
$$
\end{theorem}

%
% BibTeX users please use
\bibliographystyle{plain}
\bibliography{deltasobol}


%%%%%%%%%%%%%%%%%%%%%%%%%%%%%%%%%%%%%%%%%%%%%%%%%%%%%%%%%%%%%%%%%%%%%%

\end{document}
