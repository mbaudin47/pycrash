% Michaël Baudin, 2017
\documentclass{article}

% Copyright (C) 2012 - 2013 - EDF R&D - Michael Baudin

\setbeameroption{hide notes}
%\setbeameroption{show notes}
%\setbeameroption{show only notes}
%\mode<presentation>{\usetheme{EDF09}}

% Configuration beamer
\usetheme{Montpellier}
\setbeamertemplate{navigation symbols}{} % Remove navigation
\useoutertheme{infolines}
\setbeamertemplate{theorems}[numbered] 
% Utilise des fonts serif, pour �viter les pb de fonte
\usefonttheme{serif} 
\setbeamertemplate{caption}[numbered]


\usepackage[utf8]{inputenc}
\usepackage[T1]{fontenc}


\usepackage[french]{babel}
\uselanguage{French}
\languagepath{French}

% Scilab macros
\newcommand{\sciobj}[1]{\texttt{#1}}
\newcommand{\scifile}[1]{\texttt{#1}}

% Python macros
\newcommand{\pyobj}[1]{\textcolor{blue}{\texttt{#1}}}

\def\RR{\mathbb{R}}
\def\NN{\mathbb{N}}
\def\ZZ{\mathbb{Z}}
\def\bx{{\bf x}}

% To highlight source code
\usepackage{listingsutf8}
\lstset{inputencoding=utf8/latin1}

\definecolor{darkgreen}{rgb}{0,0.5,0}
\definecolor{violet}{rgb}{0.5,0,1}
\lstset{
  % general command to set parameter(s)
   basicstyle=\scriptsize\ttfamily, %
   keywordstyle=\color{violet}\bfseries,%
   commentstyle=\color{darkgreen}\bfseries,%
   showspaces=false,%
   stringstyle=\color{red}\bfseries
}

\hypersetup{
    %bookmarks=true,         % show bookmarks bar?
    %unicode=false,          % non-Latin characters in Acrobat�s bookmarks
    %pdftoolbar=true,        % show Acrobat�s toolbar?
    %pdfmenubar=true,        % show Acrobat�s menu?
    %pdffitwindow=false,     % window fit to page when opened
    %pdfstartview={FitH},    % fits the width of the page to the window
    %pdftitle={My title},    % title
    %pdfauthor={Author},     % author
    %pdfsubject={Subject},   % subject of the document
    %pdfcreator={Creator},   % creator of the document
    %pdfproducer={Producer}, % producer of the document
    %pdfkeywords={keyword1} {key2} {key3}, % list of keywords
    %pdfnewwindow=true,      % links in new window
    colorlinks=true,       % false: boxed links; true: colored links
    %linkcolor=red,          % color of internal links (change box color with linkbordercolor)
    %citecolor=green,        % color of links to bibliography
    %filecolor=magenta,      % color of file links
    urlcolor=blue           % color of external links
}

%%%%%%%%%%%%%%%%%%%%%%%%%%%%%%%%%%%%%%%%%%%%%%%%%%%%%%%%%%%%%%%%%%%

\begin{document}


\title{The Delta-Method applied to Sobol' indices}

\author{Michaël Baudin}

\maketitle


\abstract{
We explore the use of the Delta-method in order to estimate 
the Sobol' indices.
}

\tableofcontents

%%%%%%%%%%%%%%%%%%%%%%%%%%%%%%%%%%%%%%%%%%%%%%%%%

\section{Convergence in distribution}

\begin{definition}
(\emph{Convergence in distribution})
Assume that $X_1,X_2,...$ is a sequence of real-valued random variables 
with cumulative distribution functions $\{F_n\}_{n\geq 0}$. 
Assume that $X$ is a real-valued random variable 
with cumulative distribution function $\{F\}_{n\geq 0}$. 
The sequence $X_n$ converges in distribution to $X$ if:
$$
\lim_{n\rightarrow \infty} F_n(X_n)=F(x).
$$
for any $x\in \RR$ at which $F$ is continuous.
In this case, we write:
$$
X_n \xrightarrow{D} X.
$$
\end{definition}

The following theorem gives an example of such convergence. 

\begin{theorem}
(\emph{Maximum of uniform random numbers})
Assume that $X_1,X_2,...$ are independent random numbers such that $X_n\sim U(0,1)$. 
Let $Y_n$ be the maximum:
$$
Y_n = \max_{1\leq i\leq n}  X_i.
$$
Therefore the sequence $Y_n$ converges in distribution to an exponential random variable, i.e.:
$$
n(1-Y_n) \xrightarrow{D} \mathcal{E}(1).
$$
\end{theorem}

\begin{proof}
\end{proof}


%%%%%%%%%%%%%%%%%%%%%%%%%%%%%%%%%%%%%%%%%%%%%%%%%

\section{Delta method}

\begin{theorem}
(\emph{Delta-method})
Assume that $X_1,X_2,...$ is a sequence of real-valued random variables 
so that 
$$
\sqrt{n} (X_n - \theta) \xrightarrow{D} \mathcal{N}(0,\sigma^2).
$$
Assume that $g$ is a real function. 
Let $\theta\in\RR$ and suppose that $g'(\theta)$ exists and that $g'(\theta)\neq 0$. 
Therefore, 
$$
\sqrt{n} (g(X_n) - g(\theta)) \xrightarrow{D} \mathcal{N}(0,\sigma^2 g'(\theta)^2).
$$
\end{theorem}

%
% BibTeX users please use
\bibliographystyle{plain}
\bibliography{deltasobol}


%%%%%%%%%%%%%%%%%%%%%%%%%%%%%%%%%%%%%%%%%%%%%%%%%%%%%%%%%%%%%%%%%%%%%%

\end{document}
