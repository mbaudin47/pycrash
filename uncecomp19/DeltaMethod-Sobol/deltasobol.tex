% Michaël Baudin, 2017
\documentclass{article}

\usepackage[utf8]{inputenc}
\usepackage[T1]{fontenc}

\usepackage{lmodern} % d'autres polices sont possibles

\usepackage[a4paper]{geometry}

\usepackage{authblk}
\usepackage{url}

\usepackage{graphicx}
\DeclareGraphicsExtensions{.jpg,.pdf,.png}

\newcommand{\ot}{Open TURNS}

\newcommand{\pyobj}[1]{\texttt{#1}}

\usepackage{listings}

\usepackage{amsmath}
\usepackage{amsthm}

\usepackage[numbers]{natbib}

% Double barred letters
\usepackage{amssymb}
\newcommand{\RR}{\mathbb{R}}
\newcommand{\bx}{{\bf x}}
\newcommand{\by}{{\bf y}}
\newcommand{\bg}{{\bf g}}
\newcommand{\bX}{{\bf X}}
\newcommand{\bzero}{{\bf 0}}
\newcommand{\btheta}{\boldsymbol{\theta}}

\newtheorem{definition}{Definition}
\newtheorem{example}{Example}
\newtheorem{theorem}{Theorem}
\newtheorem{remark}{Remark}


%%%%%%%%%%%%%%%%%%%%%%%%%%%%%%%%%%%%%%%%%%%%%%%%%%%%%%%%%%%%%%%%%%%

\begin{document}


\title{The Delta-Method applied to Sobol' indices}

\author{Michaël Baudin}

\maketitle


\abstract{
We explore the use of the Delta-method in order to estimate 
the Sobol' indices.
}

\tableofcontents

%%%%%%%%%%%%%%%%%%%%%%%%%%%%%%%%%%%%%%%%%%%%%%%%%

\section{Convergence in distribution}

\begin{definition}
(\emph{Convergence in distribution})
Assume that $X_1,X_2,...$ is a sequence of real-valued random variables 
with cumulative distribution functions $\{F_n\}_{n\geq 0}$. 
Assume that $X$ is a real-valued random variable 
with cumulative distribution function $\{F\}_{n\geq 0}$. 
The sequence $X_n$ converges in distribution to $X$ if:
$$
\lim_{n\rightarrow \infty} F_n(X_n)=F(x).
$$
for any $x\in \RR$ at which $F$ is continuous.
In this case, we write:
$$
X_n \xrightarrow{D} X.
$$
\end{definition}

The following theorem gives an example of such convergence. 

\begin{theorem}
(\emph{Maximum of uniform random numbers})
Assume that $X_1,X_2,...$ are independent random numbers such that $X_n\sim U(0,1)$. 
Let $Y_n$ be the maximum:
$$
Y_n = \max_{1\leq i\leq n}  X_i.
$$
Therefore the sequence $n(1-Y_n)$ converges in distribution to an exponential random variable, i.e.:
$$
n(1-Y_n) \xrightarrow{D} \mathcal{E}(1).
$$
\end{theorem}

\begin{proof}
By definition, the exponential distribution with rate $\lambda$ has the cumulative distribution function:
$$
F(y) = 1 - \exp(-\lambda y),
$$
for any real $y\geq 0$.
With $\lambda=1$ we must prove that:
$$
F(y) = 1 - \exp(-y),
$$
for any real $y\geq 0$. 
Let $F_n$ be the cumulative distribution function of the random variable $n(1-Y_n)$. 
By definition of the cumulative distribution function, 
\begin{align*}
F_n(z) 
&= P(n(1-Y_n)\leq z) \\
&= P(1-Y_n\leq z/n) \\
&= P(1-z/n\leq Y_n) \\
&= 1-P(Y_n\leq 1-z/n),
\end{align*}
for any $z\in\RR$. 
However, the cumulative distribution function of the maximum is: 
\begin{align*}
P(Y_n\leq y)
& = P(X_1\leq y, X_2\leq y,..., X_n\leq y) \\
& = P(X_1\leq y)P(X_2\leq y) \cdots P(X_n\leq y) \textrm{ (by independence)}\\
& = y y \cdots y \textrm{ (by definition of the C.D.F. of a uniform random variable)}\\
& = y^n,
\end{align*}
for any $y\in[0,1]$. 
Therefore, 
\begin{align*}
F_n(z) 
&= 1-(1-z/n)^n
\end{align*}
for any $z\in\RR$. 
But 
\begin{align*}
\lim_{n\rightarrow \infty} (1+z/n)^n = \exp(z)
\end{align*}
for any $z\in\RR$. 
Therefore,
\begin{align*}
\lim_{n\rightarrow\infty} F_n(z) 
&= 1-\exp(-z)
\end{align*}
for any $z\in\RR$ which concludes the proof.
\end{proof}

%%%%%%%%%%%%%%%%%%%%%%%%%%%%%%%%%%%%%%%%%%%%%%%%%

\section{Convergence in probability}

\begin{definition}
(\emph{Convergence in probability})
Assume that $X_1,X_2,...$ is a sequence of real-valued random variables. 
The sequence $X_n$ converges in probability to $X$ if, for any $\epsilon>0$:
$$
\lim_{n\rightarrow \infty} P(|X_n-X|>\epsilon)=0.
$$
In this case, we write:
$$
X_n \xrightarrow{P} X.
$$
\end{definition}

\begin{theorem}
\label{theo-doubleproba}
Assume that $X$ and $Y$ are random variables. 
Let $y\in\RR$ and let $\epsilon>0$. 
Therefore
$$
P(Y\leq y) \leq P(X\leq y+\epsilon) + P(|Y-X|> \epsilon)
$$
\end{theorem}

\begin{proof}
\begin{align*}
P(Y\leq y)
& = P(Y\leq y,X\leq y+\epsilon)+P(Y\leq y,X> y+\epsilon).
\end{align*}
But $P(Y\leq y,X\leq y+\epsilon) \leq P(X\leq y+\epsilon)$, which implies :
\begin{align*}
P(Y\leq y)
& \leq P(X\leq y+\epsilon)+P(Y\leq y,X> y+\epsilon) \\
& \leq P(X\leq y+\epsilon)+P(Y-X\leq y-X, y+\epsilon<X) \\
& \leq P(X\leq y+\epsilon)+P(Y-X\leq y-X, y-X <-\epsilon).
\end{align*}
Moreover, we have $P(Y-X\leq y-X, y-X <-\epsilon)\leq P(Y-X<-\epsilon)$ which implies : 
\begin{align*}
P(Y\leq y)
& \leq P(X\leq y+\epsilon)+P(Y-X<-\epsilon).
\end{align*}
By definition of a probability, we have $P(Y-X>\epsilon)\geq 0$. 
Therefore, 
\begin{align*}
P(Y\leq y)
& \leq P(X\leq y+\epsilon)+P(Y-X<-\epsilon) + P(Y-X>\epsilon) \\
& \leq P(X\leq y+\epsilon)+P(|Y-X|<\epsilon) 
\end{align*}
which concludes the proof.
\end{proof}

\begin{theorem}
(\emph{Convergence in probability implies convergence in distribution})
Assume that $X_1,X_2,...$ is a sequence of real-valued random variables. 
If 
$$
X_n \xrightarrow{P} X
$$
then
$$
X_n \xrightarrow{D} X.
$$
\end{theorem}

\begin{proof}
TODO
\end{proof}

%%%%%%%%%%%%%%%%%%%%%%%%%%%%%%%%%%%%%%%%%%%%%%%%%

\section{Delta method}

\begin{theorem}
(\emph{Slutky's theorem})
Let $X_n$ and $Y_n$ be two sequences of real random variables such that:
$$
X_n \xrightarrow{D} X \quad \textrm{ and } \quad Y_n \xrightarrow{D} c,
$$
where $c$ is a non-random constant. 
Therefore, 
$$
X_n Y_n \xrightarrow{D} c X.
$$
\end{theorem}

\begin{proof}
TODO
\end{proof}

%%%%%%%%%%%%%%%%%%%%%%%%%%%%%%%%%%%%%%%%%%%

\begin{theorem}
(\emph{Delta-method})
Assume that $X_1,X_2,...$ is a sequence of real-valued random variables 
so that 
\begin{align}
\label{eq-delmeth1}
\sqrt{n} (X_n - \theta) \xrightarrow{D} \mathcal{N}(0,\sigma^2).
\end{align}
Assume that $g$ is a real function. 
Let $\theta\in\RR$. 
Suppose that $g\in C^1(\RR)$ and that $g'(\theta)\neq 0$. 
Therefore, 
$$
\sqrt{n} (g(X_n) - g(\theta)) \xrightarrow{D} \mathcal{N}(0,\sigma^2 g'(\theta)^2).
$$
\end{theorem}

\begin{proof}
The Taylor expansion of $g$ at the point $\theta$ implies that there exists a $\tilde{\theta}$ between 
$X_n$ and $\theta$ such that:
$$
g(X_n) = g(\theta) + g'(\tilde{\theta})(X_n-\theta) + O((X_n-\theta)^2), \quad X_n \rightarrow \theta.
$$
This implies:
$$
\sqrt{n} (g(X_n) - g(\theta)) = g'(\theta) \sqrt{n} (X_n-\theta) + O((X_n-\theta)^2), \quad X_n \rightarrow \theta.
$$
Let $Y_n$ and $Z_n$ be the sequences of random variables defined by the equations:
$$
Y_n = g'(\theta) \sqrt{n} (X_n-\theta) \textrm{ and } Z_n = O((X_n-\theta)^2).
$$
The properties of the gaussian distribution and the equation \ref{eq-delmeth1} imply:
$$
Y_n =g'(\theta) \sqrt{n} (X_n - \theta) \xrightarrow{D} \mathcal{N}(0,\sigma^2 g'(\theta)^2).
$$
Furthermore, 
$$
Z_n = O((X_n-\theta)^2)  \xrightarrow{D} 0.
$$
In order to conclude the proof, we apply Slutky's theorem to $Y_n$ and $Z_n$. 
\end{proof}

\section{References}

Most of the results in this document can be found in \cite{Vaart2000}.

%
% BibTeX users please use
\bibliographystyle{plain}
\bibliography{deltasobol}


%%%%%%%%%%%%%%%%%%%%%%%%%%%%%%%%%%%%%%%%%%%%%%%%%%%%%%%%%%%%%%%%%%%%%%

\end{document}
